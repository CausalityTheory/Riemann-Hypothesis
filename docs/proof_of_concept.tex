\documentclass{article}
\usepackage{amsmath, amssymb}

\title{Iterative Spectral Approach to the Riemann Hypothesis}
\author{Your Name}

\begin{document}

\maketitle

\begin{abstract}
This document explores an iterative spectral operator approach to the Riemann Hypothesis, aiming to understand the distribution of the Zeta function's non-trivial zeroes through eigenvalues of a spectral operator.
\end{abstract}

\section{Introduction}
The Riemann Hypothesis states that all non-trivial zeroes of the Zeta function $\zeta(s)$ have $\Re(s) = 1/2$. This paper develops a spectral approach to analyze these zeroes iteratively.

\section{Riemann Zeta Function}
The Zeta function is:
\[
\zeta(s) = \sum_{n=1}^{\infty} \frac{1}{n^s}, \quad \text{for } \Re(s) > 1
\]
Through analytic continuation, we extend its domain to the complex plane, except for $s = 1$. The goal is to show that the non-trivial zeroes lie on the critical line $\Re(s) = 1/2$.

\section{Spectral Operators}
We define a spectral operator $H(s)$:
\[
H(s) = \frac{1}{2}s(s-1)\pi^{-s/2}\Gamma\left(\frac{s}{2}\right)\zeta(s)
\]
This operator is hypothesized to have the zeroes of $\zeta(s)$ as eigenvalues, thus providing a spectral framework for the Hypothesis.
\end{document}
